\documentclass[12pt]{article}
\usepackage{graphicx}
\usepackage{wrapfig}
\usepackage{subfigure}
\usepackage{multirow}
\usepackage{hyperref}
\usepackage{amsmath}
\usepackage{amssymb}
\usepackage{ngerman}
\usepackage[ansinew]{inputenc}
\usepackage[left=2cm,top=1cm]{geometry}

% vector graphics test
\usepackage{color}
\usepackage{transparent}
\graphicspath{{graphs/}}





\begin{document}
	\pagestyle{empty}
	\textasciitilde

\begin{titlepage}
    \centering
	\huge{Astronomisches Praktikum: Quasare}\\
	\bigskip
    \large{Versuch 2}\\
    \huge{Jan R\"{o}der \& Julia Lienert}
\end{titlepage}

%headline!!!


\tableofcontents
\pagebreak

\section{Introduction}

\section{Die Leuchtkraftentfernung}

Die Leuchtkraftentfernung ist definiert als:
\begin{equation}\label{eq:D_L}
	D_L=\frac{c}{H_0q_0^2}\left( q_0z+(q_0-1)\left[  \sqrt{1+2q_0z} -1 \right]  \right)
\end{equation}
Mit
\begin{align*}
	H_0 &= 50\text{\,km\,s}^{-1}\,\text{Mpc}^{-1} \\
	q_0 &= 0.5
\end{align*}
Und
\begin{align*}
	z(\text{S5 0014+81}) &= 3.4 \\
	V(\text{S5 0014+81}) &= 16\,\text{mag} \\
	z(\text{3C 273}) &= 0.158 
\end{align*}

\subsection{Aufgabe 1}

F\"{u}r die Leuchtkraftentfernung von S5 0014+81 setzt man die gegebene Rotverschiebung in Formel \ref{eq:D_L} ein. Man erh\"{a}lt $D_L\simeq27600\,$Mpc. \\
Die absolute Helligkeit berechnet sich nach dem Entfernungsmodul:
\begin{equation} \label{eq:dist_mod}
	M=V-5 \log\left(\frac{D_L}{10\,\text{pc}}\right)
\end{equation}
Damit ergeben sich $\text{M} = -31.21\,$mag. Die Umrechnung in Sonnenleuchtkr\"{a}fte erfolgt mit
\begin{equation}
	\frac{L}{L_\odot} = 10^{-0.4\,(M-M_\odot)} = 2.59\cdot 10^{14}
\end{equation}
Eine normale Galaxie hat eine absolute Helligkeit von $-20\,$mag, bzw. $8.55\cdot 10^9\,\text{L}_\odot$. Ein Quasar ist also rund $10^5$ mal heller als eine normale Galaxie.
 
\subsection{Aufgabe 2}

Wenn die Helligkeitsschwankungen in der Gr\"{o}"senordnung eines Jahres liegt, und sich das Licht auch durch das vorhandene Medium mit etwa $c$ ausbreitet, dann ist die Gr\"{o}"se in etwa gegeben durch $x = ct$. Setzt man ein Jahr ein, so ist die emittierende Region folglich ungef\"{a}hr ein Lichtjahr gro"s (fast 63200\,AE oder gut 0.3\,pc). Eine typische Galaxie misst ca. 100000 Lichtjahre im Durchmesser, das Sonnensystem ca. 65\,AE. Die emittierende Region liegt also, was die Gr"{o}"senordnungen betrifft, zwar 6 unterhalb einer Galaxie, aber 3 oberhalb des Sonnensystems.

\subsection{Aufgabe 3}

Es ist nicht m\"{o}glich, innerhalb eines Lichtjahres durch normale Sterne eine solche Leuchtkraft zu erzeugen. Das einzige bekannte Objekt, das eine derartige Energiequelle darstellen kann, ist ein aktiver Galaxienkern (AGN). Im Zentrum befindet sich ein schwarzes Loch, eingeschlossen in einen Molek\"{u}ltorus, welcher eine Akkretionsscheibe speist. Wenn Materie auf das schwarze Loch zuf\"{a}llt, heizt sie sich auf, und gravitative Bindungsenergie wird umgesetzt in Strahlungsenergie (Radio- bis R\"{o}ntgenstrahlung). \\
Nur 10\% der AGNs sind radiolaut; man nennt diesen Anteil der AGNs oder QSOs (quasi-stellar objects) Quasare (quasi-stellar radio source).  \\
W\"{a}hrend man Quasare im optischen Bereich mit einer punktf\"{o}rmigen Quelle am Himmel identifiziert, teilt sich die Struktur im Radiobereich in eine kompakte, zentrale Quelle sowie meist zwei keulenf\"{o}rmige Jets auf. 

\subsection{Aufgabe 4}

Die Radiokarte im Anhang enth\"{a}lt mehrere Momentaufnahmen des Quasars 3C 273 und eines Knotens, der sich vom Rest der Quelle entfernt. Die Geschwindigkeit dieses Knotens hat einen Anteil in s\"{u}dlicher und einen in westlicher Richtung:
\begin{align*}
	v_S&=0.225\,\frac{\text{mas}}{\text{yr}}\\
	v_W&=0.901\,\frac{\text{mas}}{\text{yr}}
\end{align*}
Daraus ergibt sich die Gesamtwinkelgeschwindigkeit von v$_\text{SW}=0.929\,$mas/yr bzw. v$_\text{SW}=0.929$\,\,$\mathring{}$/yr. 

\subsection{Aufgabe 5}

Mit
\begin{equation*}
	x=D_L\,\tan(\alpha)
\end{equation*}
erh\"{a}lt man eine Geschwindigkeit von v$_\text{SW}=14.438\,c$. $D_L$ wurde wieder mit der gegebenen Rotverschiebung sowie Formel \ref{eq:D_L} berechnet, und zwar zu $D_L\simeq982.1$\,Mpc.


\section{Scheinbare \"{U}berlichtgeschwindigkeit}

\section{Aufgabe 1}

Die aus der Radiokarte gemessene Geschwindigkeit ist deutlich gr\"{o}"ser als die Lichtgeschwindigkeit. 















\pagebreak



\begin{thebibliography}{2}
%\bibitem{Eichler} H. J. Eichler, H.-D. Kronfeldt, J. Sahm, \textit{Das Neue Physikalische Grundpraktikum}, Springer-Verlag, Berlin-Heidelberg, 2001.
%\bibitem{Kuchling} H. Kuchling, \textit{Taschenbuch der Physik, 21. Auflage}, Fachbuchverlag Leipzig, 2014.
\end{thebibliography}

\end{document}